\documentclass{report}

% set font encoding for PDFLaTeX, XeLaTeX, or LuaTeX
\usepackage{ifxetex,ifluatex}

\if\ifxetex T\else\ifluatex T\else F\fi\fi T%
  \usepackage{fontspec}
\else
  \usepackage[T1]{fontenc}
  \usepackage[utf8]{inputenc}
  \usepackage{lmodern}
\fi

\usepackage{amsmath}
\usepackage{amssymb}
\usepackage{amsthm}
\usepackage{bm}
\usepackage{bbm}
\usepackage{mathtools}
\usepackage{physics}

\usepackage{enumitem}
\usepackage{multicol}
\usepackage{graphicx}
\usepackage[dvipsnames]{xcolor}

\usepackage{hyperref}
\hypersetup{colorlinks=true,}

\usepackage[parfill]{parskip}
\usepackage{lipsum}
\usepackage[export]{adjustbox}
\usepackage{listings}

\usepackage{xparse} 
\usepackage{subfig} 
\usepackage{xparse} 
\usepackage{float}

\usepackage[sorting=none]{biblatex} 

%%%%%This is an image table command, can likely be deleted
\newcommand{\subf}[2]{

%
{\small 
\begin{tabular}
  [t]{@{}c@{}} #1\ 
  \#2 
\end{tabular}
}

%
} 

\makeatletter
\renewcommand*\env@matrix[1][c]{\hskip -\arraycolsep
  \let\@ifnextchar\new@ifnextchar
  \array{*\c@MaxMatrixCols #1}}
\makeatother
%%%%%% Tensor Product
\NewDocumentCommand{\tens}{e{_^}}{ 
\mathbin{\mathop{\otimes}\displaylimits \IfValueT{#1}{_{#1}} \IfValueT{#2}{^{#2}} }}
%%%%%% Add \R Reals
\newcommand{\R}{\mathbb{R}} 
\newcommand{\N}{\mathbb{N}} 
\newcommand{\Z}{\mathbb{Z}} 
%%%%%% Missing citation warn
\newcommand{\CITEMISSING}{\colorbox{BurntOrange}{CITE ME}}
%%%%%% Add \theorem float
\newtheorem{theorem}{Theorem}
%%%%%% Add \definition float
\theoremstyle{definition} 
\newtheorem{definition}{Definition}[section]
%%%%%%%%%%%%%%%%%%%%%%%%%%%%%%%%%%%%%%%%%%%%%%%%%%%%%%%%%%%%%%%%%%%%%%%%%%%%%%%%%%%%%%%%%%%%%%%%%%%%%%%%%%%%%%%%%%%%%%%%%%
%%%%%Uncomment to add citation library 
\bibliography{lib} 
\title{Notes}
\author{David Helekal}
\begin{document}
\maketitle
\newpage
\tableofcontents
\newpage
\chapter{Introduction}
In epidemiology, it is often desired to be able to reconstruct the history of a pathogen population and it's structure. The problem of reconstructing the hisotry of a pathogen population can be tackles using phylodynamics. Phylodynamics utilises genomic data to assemble phylogenies, which are then used to infer the population size history. This is possible by viewing a phylogeny as a realisation of a coalescent process, wit appropriately rescaled time. This claim can be justified by viewing the coalescent as a Moran model, run backwards in time with the time rate equal to the population size \cite{griffiths_sampling_1994}.\\
Within this report we will first introduce the coalescent process for phylodynamic inference, review it's inhomogenous generalisation, and finally introduce the main result of this work, a new model capable of doing local phylodynamic inference, i.e. on a subset of the whole population capable of detecting and modelling clonal expansions.\\
Clonal expansions are a process in which a particular subsest of a given bacterial strain undergoes explosive population growth that can be traced back to a particular individual \cite{smith_how_1993}. The presence of clonal expansions in bacterial populations have been of long-standing interest and is implicated in epidemic processes, were an outbreak can be traced to a single ancestor \cite{smith_how_1993,spratt_displaying_2004,fraser_neutral_2005,ledda_re-emergence_2017}. This often happens when a particular strain or individual obtains a variant of a particular gene that confers evolutionary advantange, for example, antibiotic resistance \cite{holden_genomic_2013,hsu_evolutionary_2015, ledda_re-emergence_2017}.\\
The presence of clonal expansions leaves an imprint in the overall population structure of a given bacterial strain, the particular topology associated with this often being referred to as star-like \cite{smith_how_1993,spratt_displaying_2004}.
The problem of detecting hidden population structure corresponding to clonal expansions has become a problem of interest in epidemiology and outbreak surveillance \cite{volz_identification_nodate}.\\
While methods to detect inhomogeneities in the population structure and size have been of interest since the early days of genetic sequencing \cite{smith_how_1993,spratt_displaying_2004}, the interest in the problem increased with whole genome sequencing becoming more accessible and affordable \cite{holden_genomic_2013,dearlove_measuring_2015,eldholm_four_2015}.\\
Despite the problems of inferring population size from a genealogy and detecting heterogeneities in the population size of the entire population being intrinsically tied, all but one method \cite{volz_identification_nodate}, to our knowledge, rely either on manual detection or indirect detection. We aim to propose a simulation for the formation of clonal expansions in genealogy using the structured coalescent process, and devise a fully bayesian method for joint estimation and detection of relative population size and clonal expansions.
\chapter{Methods}
\section{Coalescent Preliminaries}
\chapter{Results}
\chapter{Discussion}
\chapter{Bibliography}
\printbibliography
\end{document}
