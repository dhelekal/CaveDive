\documentclass{report}

% set font encoding for PDFLaTeX, XeLaTeX, or LuaTeX
\usepackage{ifxetex,ifluatex}

\if\ifxetex T\else\ifluatex T\else F\fi\fi T%
  \usepackage{fontspec}
\else
  \usepackage[T1]{fontenc}
  \usepackage[utf8]{inputenc}
  \usepackage{lmodern}
\fi

\usepackage{amsmath}
\usepackage{amssymb}
\usepackage{amsthm}
\usepackage{bm}
\usepackage{bbm}
\usepackage{mathtools}
\usepackage{physics}

\usepackage{enumitem}
\usepackage{multicol}
\usepackage{graphicx}
\usepackage[dvipsnames]{xcolor}

\usepackage{hyperref}
\hypersetup{colorlinks=true,}

\usepackage[parfill]{parskip}
\usepackage{lipsum}
\usepackage[export]{adjustbox}
\usepackage{listings}

\usepackage{xparse} 
\usepackage{subfig} 
\usepackage{xparse} 
\usepackage{float}

\usepackage[sorting=none]{biblatex} 

%%%%%This is an image table command, can likely be deleted
\newcommand{\subf}[2]{

%
{\small 
\begin{tabular}
  [t]{@{}c@{}} #1\ 
  \#2 
\end{tabular}
}

%
} 

\makeatletter
\renewcommand*\env@matrix[1][c]{\hskip -\arraycolsep
  \let\@ifnextchar\new@ifnextchar
  \array{*\c@MaxMatrixCols #1}}
\makeatother
%%%%%% Tensor Product
\NewDocumentCommand{\tens}{e{_^}}{ 
\mathbin{\mathop{\otimes}\displaylimits \IfValueT{#1}{_{#1}} \IfValueT{#2}{^{#2}} }}
%%%%%% Add \R Reals
\newcommand{\R}{\mathbb{R}} 
\newcommand{\N}{\mathbb{N}} 
\newcommand{\Z}{\mathbb{Z}} 
%%%%%% Missing citation warn
\newcommand{\CITEMISSING}{\colorbox{BurntOrange}{CITE ME}}
%%%%%% Add \theorem float
\newtheorem{theorem}{Theorem}
%%%%%% Add \definition float
\theoremstyle{definition} 
\newtheorem{definition}{Definition}[section]
%%%%%%%%%%%%%%%%%%%%%%%%%%%%%%%%%%%%%%%%%%%%%%%%%%%%%%%%%%%%%%%%%%%%%%%%%%%%%%%%%%%%%%%%%%%%%%%%%%%%%%%%%%%%%%%%%%%%%%%%%%
%%%%%Uncomment to add citation library 
\bibliography{lib}
\title{Model Description}
\author{David Helekal}

\begin{document}
\chapter{Model}
\section{Coalescent with Local Population Structure}
We now present novel local phylodynamic model.
In order to help illustrate the concepts behind this model, consider the following scenario:
Suppose there is a parent population of bacteria of interest evolving through time. At any point in time, a particular individual within this population gains a significant evolutionary advantage. This advantage enables the individual and its progeny to undergo a rapid clonal expansion as it gains the ability to colonise a particular niche, eventually reaching a constant population size equilibrium. We will associate the clade corresponding to the clonal expansion with a particular colouring. 
The clonal expansion process may repeat throughout time, giving rise to several clades. The clonal expansions may for example correspond to process such a bacterial strain gaining antibiotic resistance, or being newly introduced into a hospital.

In general, we will assume that the clonal expansion process happens relatively rarely. We will also assume the parent population is at eqillibrium. As under this scenario populations reach a constant population size in the limit this assumption can be interpreted that the clonal expansion that gave rise to the parent population has happened a very long time ago.

At present a researcher collects a random set of $N$ samples from the population of interest, containing at least some of the clades produced by the clonal expansion process described above. Each sample belongs to one of the $k$ clonal expansion clades with multinomial probability $\theta_i$, $\pmb{\theta} = [\theta_1, ..., \theta_i, ..., \theta_{k-1}, \theta_{k}]$.

The researcher then sequences the genomes of the samples and infers the corresponding phylogenetic using any of the popular available methods.
The resulting phylogeny can be viewed as a realisation of the following backwards time process. 

Select $K$ clade colours, and an associated clade sampling membership probability vector $\pmb{\theta}$. For each of the $N$ samples $s_i$ forming the leaves of the phylogeny, assign a colouring $c_i$. The samples of identical colour then coalesce with each other with rate governed by a colour specific population size function $\alpha_i(t)$. As it is assumed each colour is formed by a clonal expansion at time $t^{Div}_{i}$, $\alpha_i(t)$ vanishes to zero at the time of the expansion, and as such all coalescence within clade of colour $c_i$ happens almost surely before $t^{Div}_{i}$. Upon reaching the time of clonal expansion the MRCA then chnages colour to that of another clade extant at the time. The clade of the colour the MRCA changes to will be referred to as the parent clade. The MRCA is then added to the parent clade as a leaf. 
The process continues until all only one individual remains.

The coalescent process given leaf colouring and growth function parameters described above is characterised mathematically in the following section.
%%%%%%%%%%%%%%%%%%%%%%%%%%%%%%%%%%%%%%%%%%%%%%%%%%%%%%%%%%%%
%%%%%%%%%%%%%%%%%%%%%%%%%%%%%%%%%%%%%%%%%%%%%%%%%%%%%%%%%%%%
%%%%%%%%%%%%%%%%%%%%%%%%%%%%%%%%%%%%%%%%%%%%%%%%%%%%%%%%%%%%
%%%%%%%%%%%%%%%%%%%%%%%%%%%%%%%%%%%%%%%%%%%%%%%%%%%%%%%%%%%%
\subsection{Model}
\section{Preliminaries}
A given genealogy $\mathbf{g}=(V_\mathbf{g}, E_\mathbf{g}, t_\mathbf{g})$ is an incomplete, empirical sample of the underlying process.\\
It consists of nodes $V_\mathbf{g}$, directed edges $E_\mathbf{g}$, and node labels $t_\mathbf{g}$ corresponding to event times.\\
The genealogy $\mathbf{g}$ shall be indexed by an index set $S=1\leq i \leq N\subset \N$, with $Y\subset S$ corresponding to coalescent events and $I\subset S$ corresponding to sampling events.\\
For convenience, assume that all edges are in the forwards time direction, i.e.: 
\begin{gather*}
\forall k,l \in S: (k,l)\in E_\mathbf{g} \Rightarrow t_k<t_l
\end{gather*}
Furthermore, all event times are ordered in descending (backwards) time order, with the first event corresponding the the most recent sample
\begin{gather*}
\forall k,l \in S: k<l \Rightarrow t_k > t_l
\end{gather*}
Under the assumption that $\mathbf{g}$ is a genealogy of a given sample, with each edge in $E_\mathbf{g}$ there is an associated unobserved set of individuals descending from one another. At some point along an edge from one lineage to another, the lineage can undergo a colour change, and become the most recent ancestor of a diverging clade. This event corresponds to this lineage somehow gaining advantage over other lineages, be it a bacterium gaining resistance against a drug, or a strain of a virus invading a completely susceptible population.
\begin{definition}[Multiple Lineage Coalescent]\label{def:model}
Given $M$ colours, $M$ population size functions $\mathbf{\alpha}\triangleq\{\alpha_j(t)\}_{1\leq j\leq M}$. Let $Y(t)$ be a CTMC with the state space:
\begin{gather}
  \Sigma = \left\{\mathbf{s}\in \Z_+: |\mathbf{s}|\geq1\right\}
\end{gather}
and the transition rates
\begin{gather}\label{eq:multirate}
\begin{align}
\mathbf{s}&\to\mathbf{s}-\mathbf{e_j} &\quad& \binom{s_j}{2}\alpha_j^{-1}(t)&\quad&1\leq j\leq M\\
\mathbf{s}&\to\mathbf{s}-\mathbf{e_j}+\mathbf{e_k}&\quad& \delta_{1,j}\beta s_k&\quad&1\leq j,k\leq M
\end{align}
\end{gather}
Where $\beta$ is an unknown rate.
\end{definition}
The interpretation of this model in backwards (coalescent) time is that each node corresponds to a single specific clade (colour). Nodes of the same clade coalesce at i.i.d rates, according to a clade specific growth functions, until reaching the most recent common ancestor (MRCA) of given clade. The MRCA then changes type (colour) to that of any other clade that is extant at a given time. \\
Under the hypothesis, along each of the edges from the parent of a clade MRCA to the MRCA lies a point in that characterises the time of divergence of the clade, after which the clade starts to undergo clonal expansion.
%%%%%%%%%%%%%%%%%%%%%%%%%%%%%%%%%%%%%%%%%%%%%%%%%%%%%%%%%%%%
%%%%%%%%%%%%%%%%%%%%%%%%%%%%%%%%%%%%%%%%%%%%%%%%%%%%%%%%%%%%
%%%%%%%%%%%%%%%%%%%%%%%%%%%%%%%%%%%%%%%%%%%%%%%%%%%%%%%%%%%%
%%%%%%%%%%%%%%%%%%%%%%%%%%%%%%%%%%%%%%%%%%%%%%%%%%%%%%%%%%%%
\section{Full Generative Model}
Given a population sample of $N$ individuals indexed by $F$, $X_F = \{x_i\}_{i \in F}$, first determine the number of recent clonal expansions k:
\begin{gather}
k \sim \texttt{poi}(\phi)
\end{gather}
Then simulate the $k+1$ clonal expansion probability vector $\pmb{\theta}$:
\begin{gather}
\pmb{\theta} \sim \texttt{dirichlet}(\psi), \quad \psi \in \R_+^{k+1} 
\end{gather}
Given the number of expansions $k$ and expansion probabilities $\pmb{\theta}$, partition $F$ into $k+1$ mutually disjoint subsets $\mathbf{f}=\{f_1, ... f_k, f_{k+1}\}$ with $\bigcup\limits_{i=1}^{k+1} f_{i} = F$ with the probability $P[j\in f_{i}]=\theta_i\quad \forall i,\forall j \in F$.  
Each partition corresponds to an individual colour characterising a clonal expansion. For convenience, we will set $f_{k+1}$ to be the index set of the tips corresponding to the ancestral subpopulation.

For each of the $k$ expansions $f_i$, define the parent subpopulation by $\rho_i$.

Individual subpopulations then follow the coalescent process described above in \ref{def:model}, governed by population size functions $\alpha^{-1}_{j}(t)$. To each of $\alpha^{-1}_{j}$ corresponds a set of parameters:
\begin{gather}
\begin{aligned}
T_{div_{j}}\in\R_+ \quad&\text{The time of divergence}\\
r_j\in\R_+\quad&\text{The growth rate of the subpopulation}\\
N_j\in\R_+\quad&\text{The carrying capacity}
\end{aligned}
\end{gather}
In the case of the ancestral subpopulation $f_{k+1}$, we assume that the divergence event happened a very long time ago. As by definition $\alpha(t) \rightarrow N$ as $t-T_{div}\rightarrow -\infty$, we approximate $\alpha_{k+1}(t) \approx N_{k+1}$ and as such $\alpha^{-1}_{k+1}(t) \approx 1/N_{k+1}$. The parameters $T_{div_{j}}, r_j, N_j$ are simulated from appropriate (prior) distributions.
Finally, without loss of generality, assume that divergence events are indexed in descending order by the time of divergence
\begin{gather}
i > j \Leftrightarrow T_{div_i} < T_{div_j}  
\end{gather}
%%%%%%%%%%%%%%%%%%%%%%%%%%%%%%%%%%%%%%%%%%%%%%%%%%%%%%%%%%%%
%%%%%%%%%%%%%%%%%%%%%%%%%%%%%%%%%%%%%%%%%%%%%%%%%%%%%%%%%%%%
%%%%%%%%%%%%%%%%%%%%%%%%%%%%%%%%%%%%%%%%%%%%%%%%%%%%%%%%%%%%
%%%%%%%%%%%%%%%%%%%%%%%%%%%%%%%%%%%%%%%%%%%%%%%%%%%%%%%%%%%%
\subsection{Posterior}
The posterior for the full process is:
\begin{gather}
\begin{aligned}
P(k, \mathbf{f}, \pmb{\theta}, \pmb{T_{div}}, \mathbf{r}, \mathbf{N}\mid\mathbf{g}) &\propto \mathcal{L}(\mathbf{g}\mid k, \mathbf{f}, \pmb{\rho}, \pmb{\theta}, \pmb{T_{div}}, \mathbf{r}, \mathbf{N})\\
&\cross \mathcal{L}(\mathbf{f}\mid\pmb{\theta}, k)\\
&\cross \pi(\pmb{\rho}\mid\mathbf{T_{div}})\\
&\cross \pi(\mathbf{T_{div}}, \mathbf{r}, \mathbf{N}\mid k)\\
&\cross\pi(\pmb{\theta}\mid k)\pi(k)
\end{aligned}
\end{gather}
As the model dimensionality varies with $k$, the posterior has to be integrated via RJ-MCMC. Furthermore as $\pmb{\theta}$ has to be a multinomial probability vector, it must be rescaled with every transdimensional move so that it sums to one, leading to non-unitary determinants. For inference we use the following priors:
The prior on the parental clades of individual clonal expansions
\begin{gather}
\pi(\pmb{\rho}\mid \mathbf{T_{div}}) 
\end{gather}
is assumed to be uniform probability that the clonal expansion diverged from any other subpopulation extant at the time, i.e.:
\begin{gather}
\pi(\pmb{\rho}\mid \mathbf{T_{div}}) = \prod\limits_{i=1}^{k} \mathbbm{1}_{\{x:x>i\}}(\rho_i)\frac{1}{k+1-i}  
\end{gather}
The prior on functions of effective population size can be expressed as
\begin{gather}
\pi(\mathbf{T_{div}}, \mathbf{r}, \mathbf{N}\mid k) = \pi(N_{k+1})\prod\limits_{i=1}^{k}\pi(N_i)\pi(T_{div_i})\pi(r_i)
\end{gather}
For $\pi(N_i)\quad \forall i\in\{0,...,k-1,k\}$, log-normal distributions parametrised by mean $\mu_N$ and standard deviation $\sigma_N$ provided by the decision maker are used.\\
$\pi(r_i)\quad \forall i\in\{1,...,k-1,k\}$, log-normal distributions parametrised by mean $\mu_r$ and standard deviation $\sigma_r$ provided by the decision maker are used.\\
For prior on membership probabilities
\begin{gather}
\pi(\pmb{\theta}\mid k)
\end{gather}
$k$-dimensional $\texttt{dirichlet}(\psi)$ is used, with $\psi$ supplied by the decision maker.

For prior on the number of clonal expansions
\begin{gather}
\pi(k)
\end{gather}
we use poisson distribution $\texttt{poi}(1)$
\chapter{References}
\printbibliography
\end{document}