\documentclass{report}

% set font encoding for PDFLaTeX, XeLaTeX, or LuaTeX
\usepackage{ifxetex,ifluatex}

\if\ifxetex T\else\ifluatex T\else F\fi\fi T%
  \usepackage{fontspec}
\else
  \usepackage[T1]{fontenc}
  \usepackage[utf8]{inputenc}
  \usepackage{lmodern}
\fi

\usepackage{amsmath}
\usepackage{amssymb}
\usepackage{amsthm}
\usepackage{bm}
\usepackage{bbm}
\usepackage{mathtools}
\usepackage{physics}

\usepackage{enumitem}
\usepackage{multicol}
\usepackage{graphicx}

\usepackage{hyperref}
\hypersetup{colorlinks=true,}
\usepackage[parfill]{parskip}
\usepackage{lipsum}
\usepackage[export]{adjustbox}
\usepackage{listings}

\usepackage{xparse} 
\usepackage{subfig} 
\usepackage{xparse} 
\usepackage{float}

\usepackage{biblatex} 

%%%%%This is an image table command, can likely be deleted
\newcommand{\subf}[2]{

%
{\small 
\begin{tabular}
  [t]{@{}c@{}} #1\ 
  \#2 
\end{tabular}
}

%
} 

\makeatletter
\renewcommand*\env@matrix[1][c]{\hskip -\arraycolsep
  \let\@ifnextchar\new@ifnextchar
  \array{*\c@MaxMatrixCols #1}}
\makeatother
%%%%%% Tensor Product
\NewDocumentCommand{\tens}{e{_^}}{ 
\mathbin{\mathop{\otimes}\displaylimits \IfValueT{#1}{_{#1}} \IfValueT{#2}{^{#2}} }}
%%%%%% Add \R Reals
\newcommand{\R}{\mathbb{R}} 
\newcommand{\N}{\mathbb{N}} 
\newcommand{\Z}{\mathbb{Z}} 
%%%%%% Add \theorem float
\newtheorem{theorem}{Theorem}
%%%%%% Add \definition float
\theoremstyle{definition} 
\newtheorem{definition}{Definition}[section]
%%%%%%%%%%%%%%%%%%%%%%%%%%%%%%%%%%%%%%%%%%%%%%%%%%%%%%%%%%%%%%%%%%%%%%%%%%%%%%%%%%%%%%%%%%%%%%%%%%%%%%%%%%%%%%%%%%%%%%%%%%
%%%%%Uncomment to add citation library 
\bibliography{lib} 
\title{Notes}
\author{David Helekal}

\begin{document}
\maketitle
\newpage
\tableofcontents
\newpage
%%%%%%%%%%%%%%%%%%%%%%%%%%%%%%%%%%%%%%%%%%%%%%%%%%%%%%%%%%%%
%%%%%%%%%%%%%%%%%%%%%%%%%%%%%%%%%%%%%%%%%%%%%%%%%%%%%%%%%%%%
%%%%%%%%%%%%%%%%%%%%%%%%%%%%%%%%%%%%%%%%%%%%%%%%%%%%%%%%%%%%
%%%%%%%%%%%%%%%%%%%%%%%%%%%%%%%%%%%%%%%%%%%%%%%%%%%%%%%%%%%%
\chapter{Questions}
\begin{itemize}
  \item Multi+ The second term in equation \ref{eq:multirate} (i.e. what's the bifurcation rate in the forward process) is likely not correct. Clearly this is inversely proportional to the $Neg$ of the lineage going extinct (or being birthed in the forward process), however presumably it should also be proportional to the $Neg$ of the parent lineage? The reasoning being the larger the population size of the parent lineage the likelier it is for a bifurcation event to occur.
  \item Note: there are mistakes in section \ref{section:multi}. The combination numbers need to be taken per subtree, and the bifurcation likelihood is just first iteration and needs further consideration.
\end{itemize}
%%%%%%%%%%%%%%%%%%%%%%%%%%%%%%%%%%%%%%%%%%%%%%%%%%%%%%%%%%%%
%%%%%%%%%%%%%%%%%%%%%%%%%%%%%%%%%%%%%%%%%%%%%%%%%%%%%%%%%%%%
%%%%%%%%%%%%%%%%%%%%%%%%%%%%%%%%%%%%%%%%%%%%%%%%%%%%%%%%%%%%
%%%%%%%%%%%%%%%%%%%%%%%%%%%%%%%%%%%%%%%%%%%%%%%%%%%%%%%%%%%%
\chapter{Methods}
\section{Coalescent Preliminaries}
The coalescent is a CTMC defined on the set $\{1 ... n\}$, parametrised via the coalescent rate, in our case $1/Neg(t)$, where $g$ is a scale parameter and $Neg(t)$ the population size at time $t$. 
The transition rates of the coalescent process are given by 
\begin{gather*}
\rho(j, j-1) = \binom{j}{2}\cdot\frac{1}{Neg(t)}
\end{gather*}
The waiting times in the homogenous case are exponentially distributed
\begin{gather*}
P[W_j \leq s] = 1-\exp(-s\frac{\binom{j}{2}}{Neg(t)})
\end{gather*}
Furthermore, the waiting times for individual coalescent events, conditioned on being less than the time between two consecutive sampling events $\Delta t$ are distributed as follows
\begin{gather}\label{eq:conditional}
P[W_j \leq s\mid W_j \leq \Delta t ] = \frac{P[W_j \leq s]}{P[W_j \leq \Delta t]} \quad\forall s \leq \Delta t
\end{gather}
In the inhomogenous case, the waiting times can be derived as follows:
For an inhomogenous CTMC, let $E_j(t)$ be the total exit rate from state $j$ at time $t$.
By the markov property individual exit events from a given state only depend on the state and given time, i.e. they form a time-inhomogenous poisson process.
As such the probability of no events in an interval $[t,t+s]\quad s\in \R^+$ is 
\begin{gather}
\exp(-\int_t^{t+s}E_j(\tau)d\tau) = \exp(-\int_0^{s}E_j(t+\tau)d\tau)
\end{gather}
The waiting times are defined as
\begin{gather}
W_j(t) = \inf\{s:X(t+s)\neq j \mid X(t) = j\}
\end{gather}
As such
\begin{gather}
W_j(t) > s \Rightarrow \forall \tau\in[t, t+s] X(\tau) = j
\end{gather}
Furthermore the above relation doesn't hold iff an exit event has occured in the time interval $[t,t+s]$. As such:
\begin{align*}
&P[W_j(t) > s] = P[\text{no exit events in }[t,t+s]] = \exp(-\int_0^{s}E_j(t+\tau)d\tau)\\
&P[W_j(t) < s] = 1 - \exp(-\int_0^{s}E_j(t+\tau)d\tau)
\end{align*}
In the case of phylodynamic coalescent this becomes
\begin{gather}
P[W_j(t) \leq s] = 1 - \exp(-\int_0^{s}\frac{\binom{j}{2}}{Neg(t+\tau)}d\tau)
\end{gather}
\newpage
Note, the waiting times a
re still memoryless:
\begin{gather}
P\left[W_j(t) > s+u\mid W_j(t)>s \right] = P\left[W_j(t) > s+u\mid X(s)=j\right]
\end{gather}
By markov property
\begin{gather}
P\left[W_j(t) > s+u\mid X(s)=j\right] = P\left[W_j(t+s) > u\right]
\end{gather}
%%%%%%%%%%%%%%%%%%%%%%%%%%%%%%%%%%%%%%%%%%%%%%%%%%%%%%%%%%%%
%%%%%%%%%%%%%%%%%%%%%%%%%%%%%%%%%%%%%%%%%%%%%%%%%%%%%%%%%%%%
%%%%%%%%%%%%%%%%%%%%%%%%%%%%%%%%%%%%%%%%%%%%%%%%%%%%%%%%%%%%
%%%%%%%%%%%%%%%%%%%%%%%%%%%%%%%%%%%%%%%%%%%%%%%%%%%%%%%%%%%%
\subsection{Homogenous case}
The sampling process conditioned on sampling times follows a modified gillespie scheme. In order to facilitate the computation of the likelihoods of the individual simulated trees, it is preferred to avoid rejection sampling. As such we require sampling the conditional likelihood \ref{eq:conditional}. This is achieved by inverse transform sampling.
Let:
\begin{gather}\label{eq:cond_timedep}
\begin{aligned}
  u\sim& U([0,1])\\
  T(u) : P[T(u)\leq s] &= \frac{P[T(u) \leq s]}{P[T(u) \leq \Delta t]} \quad\forall s \leq \Delta t
\end{aligned}
\end{gather}
Where $T(u)$ is assumed to be monotone increasing and invertible.
\begin{align*}
  &&P[T(u)\leq s] &= P[u\leq T^-1(s)]\\
  &\Rightarrow& P[u\leq T^{-1}(s)] &= \frac{\int_0^s\lambda\exp(-\lambda t)dt}{\int_0^{\Delta t}\lambda\exp(-\lambda t)dt}\\
  &\Rightarrow& T^{-1}(s) &= \frac{1-\exp(-\lambda s)}{1-\exp(-\lambda \Delta t)}
\end{align*}
Defining $y\triangleq T^{-1}(s)$, we obtain the transform:
\begin{gather}
T(y) = \frac{-1}{\lambda}\log[1-y(1-\exp(-\lambda\Delta t))]
\end{gather}
The corresponding pdf evaluated at $u$ is
\begin{gather}
f_{\mathbf{T}(u)}(T(u)) = \lambda \left(\frac{1}{1-\exp(\lambda\Delta t)}-u\right)
\end{gather}
\newpage
\begin{figure}[h]
  \centering
    \includegraphics[width=0.5\textwidth]{plots/Coalescent_Example.png}
    \caption{An example simulated coalescent tree}
\end{figure}
%%%%%%%%%%%%%%%%%%%%%%%%%%%%%%%%%%%%%%%%%%%%%%%%%%%%%%%%%%%%
%%%%%%%%%%%%%%%%%%%%%%%%%%%%%%%%%%%%%%%%%%%%%%%%%%%%%%%%%%%%
%%%%%%%%%%%%%%%%%%%%%%%%%%%%%%%%%%%%%%%%%%%%%%%%%%%%%%%%%%%%
%%%%%%%%%%%%%%%%%%%%%%%%%%%%%%%%%%%%%%%%%%%%%%%%%%%%%%%%%%%%
\begin{lstlisting}
f <- (sampling_times, Ne): //Sampling times in descending order
  extant_lineages <- 1
  future_lineages <- length(sampling_times)-1
  t <- sampling_times[1]
  idx <- 1
  log_lh <- 0

  while extant_lineages > 1 or future_lineages > 0:
    if extant_lineages < 2:
      idx++
      t <- sampling_times[idx]
      extant_lineages++
      future_lineages--
    else:
      delta_t <- t-sampling_times[idx+1]
      rate <- binom(extant_lineages,2)/Ne

      p_coal <- 1-exp(-rate*delta_t)
      r_c ~ U[0,1] 

      if r_c < p_coal:

        log_lh += log(p_coal)

        coalesce_lineages
        extant_lineages--

        r_w ~ U[0,1]
        w_t <- (-1/rate)*log(1-r_w*(1-exp(-rate*delta_t)))
        t <- t+w_t

        cond_lh <- rate*(1/(1-exp(-rate*delta_t)) - r_w)
        log_lh += log(cond_lh)

      else:
        log_lh += log(1-p_coal)
        idx++
        t <- sampling_times[idx]
        extant_lineages++
        future_lineages--

    return: coalescent_times, log_lh
\end{lstlisting}
%%%%%%%%%%%%%%%%%%%%%%%%%%%%%%%%%%%%%%%%%%%%%%%%%%%%%%%%%%%%
%%%%%%%%%%%%%%%%%%%%%%%%%%%%%%%%%%%%%%%%%%%%%%%%%%%%%%%%%%%%
%%%%%%%%%%%%%%%%%%%%%%%%%%%%%%%%%%%%%%%%%%%%%%%%%%%%%%%%%%%%
%%%%%%%%%%%%%%%%%%%%%%%%%%%%%%%%%%%%%%%%%%%%%%%%%%%%%%%%%%%%
\subsection{Inhomogenous Case}
In the inhomogenous case, the scheme is similar, with the key difference that the sampling times now follow a much more complex distribution. We proceed with a modified conditional sampling scheme as in \ref{eq:cond_timedep}. To obtain draws $w_j(t)$, draws from standard exponential $w_j$ are rescaled, akin to algorithm described in \cite{hein_gene_2004} Pg 98.
%%%%%%%%%%%%%%%%%%%%%%%%%%%%%%%%%%%%%%%%%%%%%%%%%%%%%%%%%%%%
%%%%%%%%%%%%%%%%%%%%%%%%%%%%%%%%%%%%%%%%%%%%%%%%%%%%%%%%%%%%
%%%%%%%%%%%%%%%%%%%%%%%%%%%%%%%%%%%%%%%%%%%%%%%%%%%%%%%%%%%%
%%%%%%%%%%%%%%%%%%%%%%%%%%%%%%%%%%%%%%%%%%%%%%%%%%%%%%%%%%%%
\subsubsection{Waiting times distribution}
Consider the time interval $[t_i, s_i]$ with $s_j = min\left\{s\in S : s>t_i\right\}$. Define $\Delta t_i \triangleq s_i-t_i$.
The probability that no coalescent events happens within this interval is 
\begin{gather*}
  P[W_j(t_i) > \Delta t_i] = \exp(-\int_0^{\Delta t_i}\frac{\binom{j}{2}}{Neg(t+\tau)}d\tau)
\end{gather*}
analogously, the probability of waiting times conditioned on that the waiting time is less than $\Delta t_i$ is:
\begin{gather}
  P[W_j(t_i) < s \mid W_j(t_i) < \Delta t_i] =\frac{1-\exp(-\int_0^{s}\frac{\binom{j}{2}}{Neg(t+\tau)}d\tau)}{1-P[W_j(t_i) > \Delta t_i]}
\end{gather}
%%%%%%%%%%%%%%%%%%%%%%%%%%%%%%%%%%%%%%%%%%%%%%%%%%%%%%%%%%%%
%%%%%%%%%%%%%%%%%%%%%%%%%%%%%%%%%%%%%%%%%%%%%%%%%%%%%%%%%%%%
%%%%%%%%%%%%%%%%%%%%%%%%%%%%%%%%%%%%%%%%%%%%%%%%%%%%%%%%%%%%
%%%%%%%%%%%%%%%%%%%%%%%%%%%%%%%%%%%%%%%%%%%%%%%%%%%%%%%%%%%%
\subsubsection{Sampling}
In order to sample $W_j(t_i)$ we proceed with an inverse transform sampling scheme, derived from the base samples $W_j$. 
First, assume $W_j$ are distributed according to 
\begin{gather}
  P[W_j < s \mid W_j < u] = \frac{1-\exp(-\frac{\binom{j}{2}}{Neg})}{1-P[W_j>u]}
\end{gather}
Where $u$ is chosen such that
\begin{gather}
P[W_j>u] = P[W_j(t_i) > \Delta t_i]
\end{gather}
Then the function
\begin{gather}
F(W_j; t_i):\quad P[F(W_j;t_i) < s \mid W_j < u] = P[W_j(t_i) < s \mid W_j(t_i) < \Delta t_i] 
\end{gather}
Is given by the inverse with respect to $s$ of
\begin{gather}
G(s; t_i) = \int_0^s \frac{Neg}{Neg(t_i+\tau)}d\tau
\end{gather}
Which exists for any biologically sensible choice of $Neg(t)$
\subsubsection{Likelihood}\label{subsection:likelihood}
Let $\left\{t_i\right\}_{i\in S\subset \N}$ denote the times of events in increasing order. Using notational convention from \cite{drummond_estimating_2002}, let $t_Y\triangleq \left\{t_i\right\}_{i\in Y}$ denote times of coalescent events and $t_I\triangleq \left\{t_i\right\}_{i\in I}$ denote times of sampling events, where $Y$, $I$ are disjoint partitions of the index set $S$ with the property that $S = Y\cup I$.
The likelihood of a particular genealogy is then given by:
\begin{gather}
\mathcal{L}\left(g\mid Neg\right) 
= \prod_{i\in S\setminus 1}\left(\mathbbm{1}_Y(i)\frac{\binom{k_{i}}{2}}{Neg(t_i)}+\mathbbm{1}_I(i)\right)
\exp(-\int_{t_{i-1}}^{t_i}\frac{\binom{k_{i}}{2}}{Neg(\tau)}d\tau)
\end{gather}
The log-likelihood is:
\begin{gather}
\log\mathcal{L}\left(g\mid Neg\right) 
= -\sum_{i\in S\setminus 1}{\int_{t_{i-1}}^{t_i}{\frac{\binom{k_{i}}{2}}{Neg(\tau)d\tau}}} + \sum_{i\in Y}{\log\frac{\binom{k_{i}}{2}}{Neg(t_i)}}
\end{gather}
%%%%%%%%%%%%%%%%%%%%%%%%%%%%%%%%%%%%%%%%%%%%%%%%%%%%%%%%%%%%
%%%%%%%%%%%%%%%%%%%%%%%%%%%%%%%%%%%%%%%%%%%%%%%%%%%%%%%%%%%%
%%%%%%%%%%%%%%%%%%%%%%%%%%%%%%%%%%%%%%%%%%%%%%%%%%%%%%%%%%%%
%%%%%%%%%%%%%%%%%%%%%%%%%%%%%%%%%%%%%%%%%%%%%%%%%%%%%%%%%%%%
\subsubsection{\textit{Skygrid} and other families of population functions}
\textit{Skygrid} is an approach introduced in \cite{gill_improving_2013}. It considers a family of time-dependent effective population size functions specified as follows. Let $T\subset\R$ be the interval under consideration in coalescent time. Consider an arbitrary population size function F:
\begin{gather}
  F:\quad T\subset\R \rightarrow \R^+
\end{gather}
Given a mutually disjoint family of time intervals $\left\{D_i\right\}_{i\in N}\subseteq T$, $D_j = [d_{j-1}, d_{j})$,  with $\bigcup\limits_{i\in N} D_i = T$, the \textit{skygrid} family of functions is then given by 
\begin{gather}
F_{skygrid}\triangleq\left\{Neg\in F\mid\forall i\in N,\quad \forall t \in D_i,\quad Neg(t) = c_i,\quad c_i\in\R^+\right\}
\end{gather}
This can be easily extended to any function $G$
\begin{gather}
 G(t)\triangleq\sum_{i\in N}{\mathbbm{1}_{D_i}(t)g_i(t)}
\end{gather}
Where $g_i(t)$ are arbitrary positive integrable functions.
Such formulation makes computation of likelihood straightforward.
\begin{gather}
\begin{aligned}
\log\mathcal{L}\left(g\mid G(t)\right) 
=& -\sum_{i\in S\setminus 1}{
\binom{k_i}{2}\sum_{j\in N}{\mathbbm{1}_{D_j}(t_{i-1})
\int_{t_{i-1}}^{\min\{d_j, t_i\}}{g^{-1}_j(\tau)d\tau}}}\\
&+\sum_{i\in Y}{\log\binom{k_i}{2}} - \sum_{i\in Y}{\log{\sum_{j\in N}{\mathbbm{1}_{D_j}(t_i)g_j(t_i)}}}
\end{aligned}
\end{gather}
%%%%%%%%%%%%%%%%%%%%%%%%%%%%%%%%%%%%%%%%%%%%%%%%%%%%%%%%%%%%
%%%%%%%%%%%%%%%%%%%%%%%%%%%%%%%%%%%%%%%%%%%%%%%%%%%%%%%%%%%%
%%%%%%%%%%%%%%%%%%%%%%%%%%%%%%%%%%%%%%%%%%%%%%%%%%%%%%%%%%%%
%%%%%%%%%%%%%%%%%%%%%%%%%%%%%%%%%%%%%%%%%%%%%%%%%%%%%%%%%%%%
\section{Inhomogenous Lineages and Outbreaks}\label{section:multi}
In this case, coalescent nodes have an added colour property, and each colour coalesces according to a colour specific, time dependent case. Nodes of non-identical colour can coalesce iff at least one of them is the last remaining node of a given colour.
Different colours correspond to different lineages, each behaving under its own growth function.\\
%%%%%%%%%%%%%%%%%%%%%%%%%%%%%%%%%%%%%%%%%%%%%%%%%%%%%%%%%%%%
%%%%%%%%%%%%%%%%%%%%%%%%%%%%%%%%%%%%%%%%%%%%%%%%%%%%%%%%%%%%
%%%%%%%%%%%%%%%%%%%%%%%%%%%%%%%%%%%%%%%%%%%%%%%%%%%%%%%%%%%%
%%%%%%%%%%%%%%%%%%%%%%%%%%%%%%%%%%%%%%%%%%%%%%%%%%%%%%%%%%%%
\subsection{Model}
A given genealogy $\mathbf{g}=(V_\mathbf{g}, E_\mathbf{g})$ is an incomplete, empirical sample of the underlying process.\\
As such, with each edge in $E_\mathbf{g}$ there is an associated unobserved set of individuals descending from one another. At some point along an edge a node of one lineage will undergo colour change and change its type to that of another lineage.\\
Given $M$ colours, $M$ population size functions $\mathbf{Neg}\triangleq\{Neg_j(t)\}_{1\leq j\leq M}$, and initial population size $N$, Let $Y(t)$ be a CTMC with the state space:
\begin{gather}
  \Sigma = \left\{\mathbf{s}\in \Z_+^{N}:|\mathbf{s}|\leq N, |\mathbf{s}|\geq1\right\}
\end{gather}.
and the transition rates
\begin{gather}\label{eq:multirate}
\begin{align}
\mathbf{s}&\to\mathbf{s}-\mathbf{e_j} &\quad& \binom{s_j}{2}Neg_j^{-1}(t)&\quad&\forall j\\
\mathbf{s}&\to\mathbf{s}-\mathbf{e_j}+\mathbf{e_k}&\quad& \delta_{1,j}s_kNeg_j^{-1}(t)&\quad&\forall j,k
\end{align}
\end{gather}
The interpretation of this model in backwards (coalescent) time is that each node corresponds to a single specific lineage (colour). Nodes of the same lineage coalesce at i.i.d rates, according to a lineage specific growth functions, until reaching the MRCA of given lineage. The MRCA then changes type (colour) to that of a different lineage. 
\subsection{Inference}
In this section, we will assume having a genealogy $\mathbf{g}=(V_\mathbf{g}, E_\mathbf{g})$, indexed by the index set $S$. The colour assigment for individual nodes will be unknown, as will the the per-colour effective population size functions $\mathbf{Neg}$, and the number of colours $M$.\\

In order to assign colourings more efficiently, we shall introduce the notion of conversion events, corresponding to transitions:
\begin{gather*}
\mathbf{s}\to\mathbf{s}-\mathbf{e_j}+\mathbf{e_k}
\end{gather*}
in equation \ref{eq:multirate}.\\

\begin{definition}[Node Colour]
Let
\begin{gather}
\mathcal{T}(j): S \mapsto \Z
\end{gather}
denote the (unknown) colour assignment for node $t_j$.
\end{definition}

\begin{definition}[Child set]
Let $\mathcal{C}(j)$ denote the child set associated with the node $t_j$
\begin{gather}
\mathcal{C}(j)\triangleq\{i\in S\mid(t_j, t_i)\in E_{\mathbf{g}},\quad t_j<t_i\}
\end{gather}
\end{definition}


\begin{definition}[Parent]
Let $P(j)$ denote the parent of the node $t_j$
\begin{gather}
P(j)\triangleq i\in S:(t_i, t_j)\in E_{\mathbf{g}},\quad t_j>t_i\}
\end{gather}
\end{definition}

\begin{definition}[Bifurcation event index]
A bifurcation index $\omega_j\in S$ satisfies the following condition
\begin{gather}
\omega\in S:\quad \mathcal{T}(\omega) \neq \mathcal{T}(P(\omega))
\end{gather}
\end{definition}

Let $\Omega$ denote the set of all bifurcation event indices associated with a given genealogy.
\begin{gather}
\Omega \triangleq \{\omega_j\}_{1\leq j\leq K}\subset S\\
\end{gather}
With the additional property that:
\begin{gather*}
\forall i<j \quad \omega_i < \omega_j
\end{gather*}

The node indexed by $\omega_i$, $t_{\omega_i}$, represents the MRCA of the lineage associated with the colouring $\mathcal{T}(\omega_i)$. Under our assumptions, there is also an unobserved (hidden) conversion event, which we will denote by $\tau^*_{\omega_i}$.\\ The event $\tau^*_{\omega_i}$ takes place somewhere along the edge $(P(\omega_j), \omega_j)$.\\
The set of all $M-1$ conversion events will be denoted $\tau^*$.
%%%%%%%%%%%%%%%%%%%%%%%%%%%%%%%%%%%%%%%%%%%%%%%%%%%%%%%%%%%%
%%%%%%%%%%%%%%%%%%%%%%%%%%%%%%%%%%%%%%%%%%%%%%%%%%%%%%%%%%%%
%%%%%%%%%%%%%%%%%%%%%%%%%%%%%%%%%%%%%%%%%%%%%%%%%%%%%%%%%%%%
%%%%%%%%%%%%%%%%%%%%%%%%%%%%%%%%%%%%%%%%%%%%%%%%%%%%%%%%%%%%
\subsubsection{Likelihood}
We have that the posterior
\begin{gather}
\mathcal{L}(\mathbf{Neg}, \tau^*, M\mid\mathbf{g}) \propto 
\mathcal{L}(\mathbf{g}\mid \tau^*, M, \mathbf{Neg})\mathcal{L}(\tau^*,M)\mathcal{L}(\mathbf{Neg})
\end{gather}
To derive the expression for the likelihood 
\begin{gather*}
\mathcal{L}(\mathbf{g}\mid \tau^*, M, \mathbf{Neg})
\end{gather*}
First define the subtrees of omega
\begin{gather}
\begin{aligned}
\mathbf{W}' &\triangleq \{W'_j\}_{1\leq j\leq K}\\
W'_j &\triangleq \{i\in S\mid t_i \text{is a descendant of } \omega_j\}
\end{aligned}
\end{gather} 
Finally, we define the subtrees corresponding to individual lineages, associated with a particular $Neg_j$, denoted by $W$:
\begin{gather}
\begin{aligned}
\mathbf{W} &\triangleq \{W_j\}_{1\leq j\leq K}\\
W_j &\triangleq W_j \setminus \bigcup_{i<j}(W_i\cup\omega_i)
\end{aligned}
\end{gather} 
Let $k_{i,j}$ denote the number of extant individuals corresponding to subtree of $W_j$ at during the time interval $[t_{i-1}, t_i]$.
The total likelihood is equal to:
\begin{gather}\label{eq:multilh}
\mathcal{L}(\mathbf{g}\mid \Omega, \mathbf{Neg}) = \prod_{j=1}^{K}\mathcal{L}(\mathbf{g}\cap W_j \mid Neg_j)\prod_{j=1}^{K}\mathcal{L}(\mathbf{g}\cap \omega_j \mid \mathbf{Neg})
\end{gather}
This can be understood as the product of the likelihoods of individual subtrees and the product of the likelihoods of the individual bifurcation events.\\
The first term in equation \ref{eq:multilh} can then be expanded as:
\begin{gather}
\prod_{j=1}^{K}\mathcal{L}(\mathbf{g}\cap W_j \mid Neg_j) 
= \prod_{j=1}^{K}\prod_{i\in S\cap W_j}\left(\mathbbm{1}_Y(i)\frac{\binom{k_{i,j}}{2}}{Neg(t_i)}+\mathbbm{1}_I(i)\right)
\exp(-\int_{t_{i-1}}^{t_i}\frac{\binom{k_{i,j}}{2}}{Neg(\tau)}d\tau)
\end{gather}
The second term:
\begin{gather}
\prod_{j=1}^{K}\mathcal{L}(\mathbf{g}\cap \omega_j \mid \mathbf{Neg})
= \prod_{i\in\Omega}\left(\frac{k_{i,j}}{Neg(t_i)}\right)
\exp(-\int_{t_{i-1}}^{t_i}\frac{k_{i,j}}{Neg(\tau)}d\tau)
\end{gather}
\subsubsection{Choice of \textit{Neg}}
\subsubsection{Inference}
Due to the variable nature of the dimensionality of the parameter space, it will be necessary to use Reversible Jump MCMC (rjMCMC) \cite{fan_reversible_2010,green_reversible_1995}.
%%%%%%%%%%%%%%%%%%%%%%%%%%%%%%%%%%%%%%%%%%%%%%%%%%%%%%%%%%%%
%%%%%%%%%%%%%%%%%%%%%%%%%%%%%%%%%%%%%%%%%%%%%%%%%%%%%%%%%%%%
%%%%%%%%%%%%%%%%%%%%%%%%%%%%%%%%%%%%%%%%%%%%%%%%%%%%%%%%%%%%
%%%%%%%%%%%%%%%%%%%%%%%%%%%%%%%%%%%%%%%%%%%%%%%%%%%%%%%%%%%%
\chapter{Results}
\section{Simulations}
\subsection{Exponetial Growth}
An example of a population under exponential growth, $Neg(t) = N*\exp(-\lambda t)$ consisting of 100 sampling events between 0 and 10 years before present has been simulated with a rate parameter $\lambda$ and final size parameter $N$ drawn from a uniform densities on intervals $[0.1, 10]$, and $[1,100]$ respectively.\\
A Metropolis-Hastings MCMC scheme was then used to infer the parameters $\lambda$ and $N$. A zero-centred laplace distribution with rate equal to one was chosen as the prior for $\lambda$, whereas for $N$ the exponential distribution with rate one was used.\\
One million iterations were used and the first One hundred thousand discarded as burn in time. To further validate the fit, both a maximum likelihood (MLE) and maximum \textit{a-posteriori} (MAP) estimates were computed and plotted against the posterior marginals inferred by the MCMC.
%%%%%%%%%%%%%%%%%%%%%%%%%%%%%%%%%%%%%%%%%%%%%%%%%%%%%%%%%%%%
%%%%%%%%%%%%%%%%%%%%%%%%%%%%%%%%%%%%%%%%%%%%%%%%%%%%%%%%%%%%
%%%%%%%%%%%%%%%%%%%%%%%%%%%%%%%%%%%%%%%%%%%%%%%%%%%%%%%%%%%%
%%%%%%%%%%%%%%%%%%%%%%%%%%%%%%%%%%%%%%%%%%%%%%%%%%%%%%%%%%%%
\begin{figure}[H]
  \centering
  \subfloat[]{\label{trace:a}\includegraphics[scale=.25]{../R/trace}}
  \centering
  \subfloat[]{\label{trace:b}\includegraphics[scale=.25]{../R/trace_zoom}}
  \centering
  \caption{Trace plots for the markov chain. Red lines denote true parameter values. MLE marked by blue lines. MAP marked by orange lines. \ref{trace:a} Shows the entire trace of the chain. \ref{trace:b} Shows the trace with the first 100000 iterations discarded}
  \label{fig:trace}
\end{figure}
%%%%%%%%%%%%%%%%%%%%%%%%%%%%%%%%%%%%%%%%%%%%%%%%%%%%'%%%%%%%%
%%%%%%%%%%%%%%%%%%%%%%%%%%%%%%%%%%%%%%%%%%%%%%%%%%%%%%%%%%%%
%%%%%%%%%%%%%%%%%%%%%%%%%%%%%%%%%%%%%%%%%%%%%%%%%%%%%%%%%%%%
%%%%%%%%%%%%%%%%%%%%%%%%%%%%%%%%%%%%%%%%%%%%%%%%%%%%%%%%%%%%
\begin{figure}[H]
  \centering
  \subfloat[]{\label{hists:a}\includegraphics[scale=.4]{../R/lambdahist}}
  \centering
  \subfloat[]{\label{hists:b}\includegraphics[scale=.4]{../R/nhist}}
  \centering
  \caption{Histograms of the posterior marginals. MLE marked by blue line. MAP marked by orange line. \ref{hists:a} $\lambda$ marginal \ref{hists:b} $N$ marginal}
  \label{fig:hists}
\end{figure}
\begin{figure}[H]
  \centering
    \includegraphics[width=0.5\textwidth]{../R/tree}
    \caption{The simulated genealogy used for this example.}
\end{figure}
%%%%%%%%%%%%%%%%%%%%%%%%%%%%%%%%%%%%%%%%%%%%%%%%%%%%%%%%%%%%
%%%%%%%%%%%%%%%%%%%%%%%%%%%%%%%%%%%%%%%%%%%%%%%%%%%%%%%%%%%%
%%%%%%%%%%%%%%%%%%%%%%%%%%%%%%%%%%%%%%%%%%%%%%%%%%%%%%%%%%%%
%%%%%%%%%%%%%%%%%%%%%%%%%%%%%%%%%%%%%%%%%%%%%%%%%%%%%%%%%%%%
\section{Previous Work}
A framework utilising sampling intensity in order to extract more information is proposed in \cite{parag_jointly_nodate}
%%%%%%%%%%%%%%%%%%%%%%%%%%%%%%%%%%%%%%%%%%%%%%%%%%%%%%%%%%%%
%%%%%%%%%%%%%%%%%%%%%%%%%%%%%%%%%%%%%%%%%%%%%%%%%%%%%%%%%%%%
%%%%%%%%%%%%%%%%%%%%%%%%%%%%%%%%%%%%%%%%%%%%%%%%%%%%%%%%%%%%
%%%%%%%%%%%%%%%%%%%%%%%%%%%%%%%%%%%%%%%%%%%%%%%%%%%%%%%%%%%%
\section{Bibliography}
\printbibliography
\end{document}
