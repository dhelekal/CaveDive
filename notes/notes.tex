\documentclass{article}

% set font encoding for PDFLaTeX, XeLaTeX, or LuaTeX
\usepackage{ifxetex,ifluatex}

\if\ifxetex T\else\ifluatex T\else F\fi\fi T%
  \usepackage{fontspec}
\else
  \usepackage[T1]{fontenc}
  \usepackage[utf8]{inputenc}
  \usepackage{lmodern}
\fi

\usepackage{amsmath}
\usepackage{amssymb}
\usepackage{amsthm}
\usepackage{bm}
\usepackage{mathtools}
\usepackage{physics}

\usepackage{enumitem}
\usepackage{multicol}
\usepackage{graphicx}
\usepackage{hyperref}
\usepackage[parfill]{parskip}
\usepackage{lipsum}
\usepackage[export]{adjustbox}
\usepackage{listings}

\usepackage{xparse} 
\usepackage{subfig} 
\usepackage{xparse} 
\usepackage{float}

\usepackage{biblatex} 

%%%%%This is an image table command, can likely be deleted
\newcommand{\subf}[2]{

%
{\small 
\begin{tabular}
  [t]{@{}c@{}} #1\ 
  \#2 
\end{tabular}
}

%
} 

\makeatletter
\renewcommand*\env@matrix[1][c]{\hskip -\arraycolsep
  \let\@ifnextchar\new@ifnextchar
  \array{*\c@MaxMatrixCols #1}}
\makeatother
%%%%%% Tensor Product
\NewDocumentCommand{\tens}{e{_^}}{ 
\mathbin{\mathop{\otimes}\displaylimits \IfValueT{#1}{_{#1}} \IfValueT{#2}{^{#2}} }}
%%%%%% Add \R Reals
\newcommand{\R}{\mathbb{R}} 
\newcommand{\N}{\mathbb{N}} 
\newcommand{\Z}{\mathbb{Z}} 
%%%%%% Add \theorem float
\newtheorem{theorem}{Theorem}
%%%%%% Add \definition float
\theoremstyle{definition} 
\newtheorem{definition}{Definition}[section]
%%%%%%%%%%%%%%%%%%%%%%%%%%%%%%%%%%%%%%%%%%%%%%%%%%%%%%%%%%%%%%%%%%%%%%%%%%%%%%%%%%%%%%%%%%%%%%%%%%%%%%%%%%%%%%%%%%%%%%%%%%
%%%%%Uncomment to add citation library 
\bibliography{lib} 
\title{Notes}
\author{David Helekal}

\begin{document}
\maketitle
\newpage
\section{Simulation}
\subsection{Coalescent Preliminaries}
The coalescent is a CTMC defined on the set $\{1 ... n\}$, parametrised via the coalescent rate, in our case $g/Ne(t)$, where $g$ is a scale parameter and $Ne(t)$ the population size at time $t$. 
The transition rates of the coalescent process are given by 
\begin{gather*}
\rho(j, j-1) = \binom{j}{2}\cdot\frac{g}{Ne(t)}
\end{gather*}
The waiting times in the homogenous case are exponentially distributed
\begin{gather*}
W_j \sim 1-\exp{-\frac{g\binom{j}{2}}{Ne(t)}}
\end{gather*}
In the inhomogenous case, the waiting times can be derived as follows:
For an inhomogenous CTMC, let $E_j(t)$ be the total exit rate from state $j$ at time $t$.
By the markov property individual exit events from a given state only depend on the state and given time, i.e. they form a time-inhomogenous poisson process.
As such the probability of no events in an interval $[t,t+s]\quad s\in \R^+$ is 
\begin{gather}
\exp{-\int_t^{t+s}E_j(\tau)d\tau} = \exp{-\int_0^{s}E_j(t+\tau)d\tau}
\end{gather}
The waiting times are defined as
\begin{gather}
W_j(t) = \inf\{s:X(t+s)\neq j \mid X(t) = j\}
\end{gather}
As such
\begin{gather}
W_j(t) > s \Rightarrow \forall \tau\in[t, t+s] X(\tau) = j
\end{gather}
Furthermore the above relation doesn't hold iff an exit event has occured in the time interval $[t,t+s]$. As such:
\begin{align*}
&P[W_j(t) > s] = P[\text{no exit events in }[t,t+s]] = \exp{-\int_0^{s}E_j(t+\tau)d\tau}\\
&P[W_j(t) < s] = 1 - \exp{-\int_0^{s}E_j(t+\tau)d\tau}
\end{align*}
In the case of phylodynamic coalescent this becomes
\begin{gather}
W_j(t) \sim 1 - \exp{-\int_0^{s}\frac{g\binom{j}{2}}{Ne(t+\tau)}d\tau}
\end{gather}
\newpage
Note, the waiting times a
re still memoryless:
\begin{gather}
P\left[W_j(t) > s+u\mid W_j(t)>s \right] = P\left[W_j(t) > s+u\mid X(s)=j\right]
\end{gather}
By markov property
\begin{gather}
P\left[W_j(t) > s+u\mid X(s)=j\right] = P\left[W_j(t+s) > u\right]
\end{gather}
\subsection{Homogenous case}
The sampling process conditioned on sampling times follows a slightly modified gillespie scheme.


\begin{figure}[h]
  \centering
    \includegraphics[width=0.5\textwidth]{plots/Coalescent_Example.png}
    \caption{An example simulated coalescent tree}
\end{figure}

\begin{lstlisting}
f <- (sampling_times, Ne): //Sampling times in descending order
  extant_lineages <- 1
  future_lineages <- length(sampling_times)-1
  t <- sampling_times[1]
  idx <- 1

  while extant_lineages > 1 or future_lineages > 0:
    if extant_lineages < 2:
      idx++
      t <- sampling_times[idx]
      extant_lineages++
      future_lineages--
    else:
      delta_t <- t-sampling_times[idx+1]
      rate <- binom(extant_lineages,2)/Ne
      w_t ~ exp(rate)

      if w_t < delta_t:
        coalesce_lineages
        extant_lineages--
        t <- t+w_t
      else:
        idx++
        t <- sampling_times[idx]
        extant_lineages++
        future_lineages--
\end{lstlisting}

\subsection{Inhomogenous Case}
In the inhomogenous case, the scheme is similar, with the key difference that the sampling times now follow a much more complex distribution. As such a sampling scheme such as rejection sampling will be required (?)
\subsection{Multistrain+Inhomogenous Case}
In this case, coalescent nodes have an added colour property, and each colour coalesces according to a colour specific, time dependent case. Nodes of non-identical colour can coalesce iff at least one of them is the last remaining node of a given colour.\\
Given $M$ colours, $M$ population size functions $\{Ne_j(t)\}_{1\leq j\leq M}$, and initial population size $N$, Let $Y(t)$ be a CTMC with the state space:
\begin{gather}
  S = \left\{\mathbf{s}\in \Z_+^{N}:|\mathbf{s}|\leq N, |\mathbf{s}|\geq1\right\}
\end{gather}
and the transition rates
\begin{gather}
\mathbf{s}\to\mathbf{s}-\mathbf{e_j} \quad \binom{s_j}{2}Ne_j(t)+\delta_{s_j, 1}Ne_j(t)\sum_{i\neq j}s_i
\end{gather}
\end{document}
